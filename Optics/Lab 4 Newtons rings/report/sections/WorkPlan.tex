\begin{center}
    \Large{\textbf{Хід роботи}}    
\end{center}

\vspace{1mm}

\textit{\textbf{Мета роботи:}} 
Ознайомлення з явищем інтерференції в тонких плівках (смуги рівної товщини)
на прикладі кілець Ньютона і з методикою інтерференційних вимірювань
кривизни скляної поверхні.
\bigbreak


В даній лабораторній роботі кільця Ньютона досліджується за допомогою
мікроскопа. На столику мікроскопа розташоване держак, на якому розміщується
досліджувана лінза з пластиною. В одному з окулярів мікроскопа 
встановлюється освітювач, що генерує пучок променів, паралельних тим, що падають в
околі спостерігача. Для монохроматизації пучка перед освітлювачем 
встановлюють фільтр. В комплект входять 7 фільтрів, що створюють монохроматичні
пучки, довжини хвилі яких наведені в таблиці 1.

\begin{table}[h]
    \centering
    \begin{tabular}{ |c|c| }
        \hline
        \textbf{Колір} & \textbf{Дожина швилі $\lambda$(нм)} \\
        \hline
        Фіолетовий & $404 \pm 10$ \\
        \hline
        Синій & $434 \pm 10$ \\
        \hline
        Блакитний  & $486 \pm 10$ \\
        \hline
        Зелений & $546 \pm 10$ \\
        \hline
        Жовтий & $586 \pm 10$ \\
        \hline
        Помаранчевий & $656 \pm 10$ \\
        \hline
        Червоний & $706 \pm 10$ \\
        \hline
    \end{tabular}
    \caption{Довжини хвиль}
\end{table}


На початку експерименту рекомендуються знайти кільця Ньютона в
білому світлі (без фільтра) і сфокусувати мікроскоп 
під своє око. Перехрестя шкал мікроскопа повинно 
проходити через центр кілець. Після цього
можна встановити фільтр і переходити
до безпосередніх вимірювань радіусу
кілець. Для вимірювань на окулярі мікроскопа
нанесено спеціальну шкалу з поділками. Ціну поділки для кожного
значення збільшення вказано в інструкції до мікроскопа.


Вимірювати радіус кілець слід від центру системи до середини кільця.
Для збільшення точності рекомендуємо після першої серії вимірів із заданим
фільтром повернути лінзу на $90^{o}$ навколо вертикальної осі і повторити виміри.
Якщо робота виконується двома студентами, то рекомендуємо провести виміри
кожному з студентів, а потім порівняти й усереднити одержані результати.

\begin{enumerate}
    \item Виміряйте радіуси темних та світлих кілець для усіх наявних фільтрів,
    Побудуйте графіки залежностей квадрату радіусів від номеру кільця.
    
    \item За графіками визначте нахил прямих і розрахуйте радіус кривизни.

    \item Оцініть діаметр плями стику лінзи зі скляною пластинкою.
\end{enumerate}