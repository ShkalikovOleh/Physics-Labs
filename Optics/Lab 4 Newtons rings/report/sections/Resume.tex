\begin{center}
    \Large{\textbf{Висновки}}    
\end{center}

\vspace{1mm}

Ми ознайомились з явищем інтерференції в тонких плівках
на прикладі кілець Ньютона і з методикою інтерференційних вимірювань
кривизни скляної поверхні.


В результаті виконання лабораторної роботи було обраховано
радіуси кривизни та діаметри плями стику лінзи зі скляною пластиною.
Результати наведені у таблиці 3. Також були обраховані абсолютні та відносні
похибки (таблиця 4). Їх можна пояснити людським фактором та
не зовсім коректним використанням обладнання.
