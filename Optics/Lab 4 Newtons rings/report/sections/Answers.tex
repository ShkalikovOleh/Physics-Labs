\begin{center}
    \Large{\textbf{Відповіді на контрольні питання}}    
\end{center}

\vspace{1mm}

\begin{enumerate}
    \item Від чого залежить кількість спостережуваних кілець?
    \bigbreak
    Від спектрального складу світла, в якому ми проводимо спостереження.
    Якщо в білому, то кілець буде відносно небагато, а якщо в монохроматичному, то їх буде дуже багато.
    Причому чим менше довжина хвилі монохроматичного світла тим більше кілець.
    
    \item Чому кільця, що спостерігаються мають райдужне забарвлення?
    \bigbreak
    Систему світлих і темних смуг отримують лише при освітленні монохроматичним світлом. При
    спостереженні в білому світлі отримують сукупність зміщених одна відносно одної смуг,
    утворених променями різних довжин хвиль, й інтерференційна картина набуває райдужного
    забарвлення.
    
    \item Чому по мірі віддалення від центру кільця розташовуються ближче один
    до одного?
    \bigbreak
    По мірі віддалення від центру кільця розташовуються ближче один
    до одного, бо змінюється кут повітряного клина.

    \item Що станеться з кільцями Ньютона, якщо проміжок між лінзою і пластинкою заповнити рідиною?
    \bigbreak
    Зміна фаз на границі рідина-скло буде відмінною від зміни фаз,
    на границі повітря-скло. Як можна побачити з формули \ref{eq:3}, якщо
    показник заломлення буде більший за показник заломлення повітря,
    то радіус кілець зменьшиться, і навпаки: якщо
    показник заломлення буде меньший за показник заломлення повітря, то радіус
    збільшиться.

    \item Чи можна при спостережені кілець Ньютона у відбитому світлі отримати в центрі не темне,
    а світле кільце? Якщо так, то сформулюйте умови, які для цього необхідні.
    \bigbreak
    При спостережені кілець Ньютона у відбитому світлі отримати в центрі не темне,
    а світле кільце можливо, якщо:
    \begin{enumerate}
        \item якщо лінза не прилягає щільно 
        \item існує ділянка, де відсутній проміжок і тому проходить світло(лінза притиснута)
    \end{enumerate}

\end{enumerate}