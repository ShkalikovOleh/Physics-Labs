\begin{center}
    \Large{\textbf{Відповіді на контрольні питання}}    
\end{center}

\vspace{1mm}

\begin{enumerate}
    \item Від чого залежить кількість спостережуваних кілець?
    \bigbreak

    
    \item Чому кільця, що спостерігаються мають райдужне забарвлення?
    \bigbreak
    
    
    \item Чому по мірі віддалення від центру кільця розташовуються ближче один
    до одного?
    \bigbreak


    \item Що станеться з кільцями Ньютона, якщо проміжок між лінзою і пластинкою заповнити рідиною?
    \bigbreak


    \item Чи можна при спостережені кілець Ньютона у відбитому світлі отримати в центрі не темне,
    а світле кільце? Якщо так, то сформулюйте умови, які для цього необхідні.
    \bigbreak


\end{enumerate}