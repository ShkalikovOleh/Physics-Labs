\begin{center}
    \Large{\textbf{Практична частина}}
\end{center}

\vspace{1mm}

У результаті проведених експериментів 
були отримані наступні дослідні дані(відстані відносно початку рейки):
\begin{table}[h] \label{table:raw}
    \centering
    \begin{tabular}{ |c|c|c| }
        \hline 
        \textbf{Відстань до екрана(см)} & \textbf{Відстань до лінзи(см)} & \textbf{Відстань до предмета(см)} \\
        \hline
        102.3 & 43.8 & \multirow{4}{*}{15} \\
        \cline{1-2}
        62.4 & 23.6 & \\
        \cline{1-2}
        36.0 & 20.6 & \\
        \cline{1-2}
        44.7 & 22.2 & \\
        \hline
    \end{tabular}
    \caption{Дослідні дані}
\end{table}


Обрахуємо відстані до лінзи за наступними формулами:
$$ g = s_l - s_o $$
$$ b = s_s - s_l $$
,де $s_l$ - відстань до лінзи від початку рейки,
$s_s$ - відстань до украна від початку рейки,
$s_o$ - відстань до предмета від початку рейки.

Згідно формули \ref{eq:1} тонкої збиральної лінзи 
та формули \ref{eq:2} для коефіцієнта збільшення лінзи обрахуємо
зазначені величини. 

Обрахуємо відносні похибки($\delta_f, \delta_V$), для цього абсолютні похибки вимірювань обчислимо за наступними формулами ($\Delta b = \Delta g = 0.1$(см))
$$ \Delta f = \sqrt{ \left( \frac{\partial{f}}{ \partial{g} } \Delta g \right)^2 +
\left( \frac{\partial{f}}{ \partial{b} } \Delta b \right)^2 } = 
\sqrt{ \frac{b^4 {\Delta g}^2}{ (b+g)^4 } + \frac{g^4 {\Delta b}^2}{ (b+g)^4 } } $$

$$ \Delta V = \sqrt{ \left( \frac{\partial{V}}{ \partial{g} } \Delta g \right)^2 +
\left( \frac{\partial{V}}{ \partial{b} } \Delta b \right)^2 } =
\sqrt{ \frac{b^2 {\Delta g}^2}{ g^4 } + \frac{{\Delta b}^2}{ g^2 } } $$

Результати обрахунків наведені у таблиці нижче:

\begin{table}[h] \label{table:fV}
    \centering
    \begin{tabular}{ |c|c|c|c|c|c|c|c| }
        \hline 
        $g$ \textbf{(cм)} & $b$ \textbf{(cм)} & $f$ \textbf{(cм)} & $V$ & $\Delta f$ \textbf{(см)} & $\Delta V$ & $\delta_f$\% & $\delta_V$\% \\
        \hline
        28.8 & 58.5 & 19.3 & 2.03 & 0.046 & 0.008 & 0.24 & 0.39 \\
        \hline
        8.6 & 38.8 & 7.04 & 4.51 & 0.067 & 0.054 & 0.95 & 1.19 \\
        \hline
        5.6 & 15.4 & 4.11 & 2.75 & 0.054 & 0.052 & 1.32 & 1.9 \\
        \hline
        7.2 & 22.5 & 5.45 & 3.125 & 0.058 & 0.045 & 1.58 & 1.46 \\
        \hline
    \end{tabular}
    \caption{Результати обрахунків}
\end{table}

%\begin{figure}[h]    
%    \centering
%    \includegraphics[width=.7\textwidth]{assets/filename }
%    \caption{Підпис}
%\end{figure}