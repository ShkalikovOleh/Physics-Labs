\begin{center}
    \Large{\textbf{Хід роботи}}    
\end{center}

\vspace{1mm}

\textit{\textbf{Мета роботи:}} Визначити фокуснi та коефiцiєнти збiльшення невiдомих лiнз.
\bigbreak

В наближеннi тонкої лiнзи вважається, що обидвi головнi площини спiвпадають i проходять через
середину лiнзи. Таким чином, користуючись для визначення фокусної вiдстанi наближенням тонкої лiнзи
ми робимо помилки на величину товщини скла лiнзи, i це обмежує необхiдну точнiсть вимiрювань. Якщо
необхiдно отримати точнiшi значення фокусних вiдстаней, то слiд зважати на вiдстань мiж головними
площинами.

Фокусну вiдстань можна визначити безпосередньо за формулою \ref{eq:1}. Для цього слiд розташувати на
рейці послiдовно освiтлювач зi шкалою(об’єкт), лiнзу та екран. Пересуваючи екран вздовж рейки, добитись
чiткого зображення (збиральна лiнза дає перевернуте зображення). За лiнiйкою на рейці визначити
вiдстанi мiж екраном та лiнзою та мiж лiнзою та освiтлювачем. За допомогою \ref{eq:1} обчислити фокусну
вiдстань. Змiнити положення екрана та повторити дослiд ще декiлька разiв. за результатами дослiду 
знайти середнє значення фокусної вiдстанi.