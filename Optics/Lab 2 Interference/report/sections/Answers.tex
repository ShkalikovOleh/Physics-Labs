\begin{center}
    \Large{\textbf{Відповіді на контрольні питання}}    
\end{center}

\vspace{1mm}

\begin{enumerate}
    \item Що таке інтерференція світлових хвиль?
    \bigbreak
    Інтерференція - явище накладання когерентних світлових хвиль при якому спостерігається перерозподіл світлового потоку в просторі.

    \item Як отримати інтерференцію світла за допомогою плоскопаралельної пластини?
    \bigbreak


    \item Яким чином утворюються уявні джерела світла? Навіщо стоїть 
    короткофокусна лінза на виході джерела світла у схемі нашої роботи?
    \bigbreak

    \item Що таке когeрентність хвиль? Просторова і часова когерентність.
    \bigbreak

    
    \item Інтерференція на тонких плівках. Чому ми бачимо різні кольори світла?
    \bigbreak

    
    \item Апертура інтерференції і пучків.
    \bigbreak


    \item Що таке інтерференція світла зі смугами рівної товщини, рівного нахилу?
    \bigbreak


\end{enumerate}