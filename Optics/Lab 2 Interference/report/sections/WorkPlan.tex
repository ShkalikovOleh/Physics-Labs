\begin{center}
    \Large{\textbf{Хід роботи}}    
\end{center}

\vspace{1mm}

\textit{\textbf{Мета роботи:}} Вивчення явища інтерференції світла, дослідження інтерференційної картини,
отриманої при відбиванні світла від товстої скляної пластини; визначення
показника заломлення скляної пластини (довжини хвилі лазера, товщини пластини).
\bigbreak


Відстані між об’єктами на оптичній лаві вимірюються за допомогою
міліметрової шкали, що нанесено на лаві, і відраховуються від рисок, 
що нанесені на штативах. Довжина хвилі гелій-неонового лазера, що 
використовується в експерименті, дорівнює 632.85 нм. Фокусна відстань
збиральної лінзи, розташованої в центрі екрана, дорівнює $f = 13 \pm 1$ мм.
Показник заломлення скла, з якого зроблена пластинка, наближено дорівнює 1.5.

\begin{enumerate}
    \item  Виміряйте радіуси не менше шести темних кілець (для кожного кільця
    треба отримати 4 значення, що визначені по взаємно-перпендикулярним
    шкалам екрану 2 і усереднити). Кільцям приписують номера N в порядку
    зростання їх радіусів. Номер $N = 1$ приписують першому темному кільцю,
    поблизу отвору в екрані.

    \item Побудуйте графік $B \Theta(L)$.
    Знайдіть кутові коефіцієнти 
    прямих, що апроксимують ці залежності.

    \item Знайдіть показник заломлення скляної пластинки, знаючи довжину хвилі
    лазера $\lambda = 632.816$ нм та товщину пластинки d, яка зазначена на ній.
    
    %%\item  Обчислити похибку цієї величини.

\end{enumerate}