\pagebreak

\begin{center}
    \Large{\textbf{Відповіді на контрольні питання}}    
\end{center}

\vspace{1mm}

\begin{enumerate}
    \item Принцип Гюйгенса-Френеля.
    \bigbreak
    Всі точки відкритої хвильової поверхні даної хвилі є
    джерелами елементарних вторинних когерентних хвиль.
    При цьому амплітуда (й інтенсивність) коливань, що збуджуються
    хвилею в будь-якій точці спостереження, визначається результатом
    інтерференції всіх вторинних хвиль, які приходять у цю точку.

    \item Що таке дифракція? Чим відрізняється дифракція Фраунгофера від дифракції
    Френеля? Критерій дифракції Фраунгофера.
    \bigbreak
    Дифракція світла - це сукупність фізичних явищ, обумовлених хвильовою природою світла і спостерігаються при його поширенні в середовищі з
    різко вираженою оптичної неоднорідністю (наприклад, при проходженні 
    через отвори в екранах, поблизу меж непрозорих тіл тощо). У більш вузькому
    сенсі під дифракцією розуміють огинання світлом різних перешкод, 
    тобто відхилення від законів геометричної оптики.

    При дифракції Френеля джерело світла, перешкода та екран для
    спостереження розташовані на порівняно невеликій відстані одне 
    від одного. Через це на перешкоду та на екран падають не паралельні
    промені, тобто не плоскі хвилі.

    При дифракції Фраунгофера на перешкоду спрямовують 
    плоскі хвилі і на екрані спостерігають результат дифракції теж 
    плоских хвиль. Плоским хвилям відповідають паралельні промені,
    тому це вид дифракції ще називають дифракцією в паралельних променях.

    Нехай на перешкоду, розміром $b$, падає хвиля з довжиною
    $\lambda$, а точка спостереження розташована на відстані $l$.
    Тоді, якщо $ \frac{b^2}{\lambda l} \ll 1 $, спостерігається
    дифракція Фраунгофера. Якщо $ \frac{b^2}{\lambda l} \simeq 1 $ - Френеля. Інакше світло
    світло поширюється за законами геометричної оптики.

    (Або ж кількість зон Френеля для дифракції Фраунгофера менша за 1)

    \item Що таке дифракційна гратка і якими параметрами вона характеризується?
    \bigbreak
    Дифракційна гратка - оптичний елемент із періодичною 
    структурою, здатний впливати на поширення світлових хвиль так,
    що енергія хвилі, яка пройшла через гратку, зосереджується 
    в певних напрямках.  Оптичні характеристики ґратки визначають три параметри:
    \begin{enumerate}
        \item період - відстань між сусідніми щілинами
        \item ширина щілини або величин
        \item загальна кількість штрихів (щілин) , яка визначається розміром ґратки
    \end{enumerate}    


    \item Як залежить ширина головного максимуму та інтенсивність від ширини щілини?
    \bigbreak
    З виразу  \ref{eq:5} , зрозуміло, що при збільшенні
    ширини щілини значення $\sin \varphi$ буду зменшуватися.
    Тобто дифракційна картина звузиться(ширина головних максимумів стане менше).
    Інтенсивність буде спадати.
    Якщо ж зменшувати $b$, то дифракційна 
    картина буде розширюватись. Інтенсивність буде збільшуватися.

    \item Як залежить ширина головного максимуму та інтенсивність від відстані між щілиною та екраном при сталій ширині щілини?
    \bigbreak
    З виразу  \ref{eq:10}, зрозуміло, що при постійному періоді $d$
    відношення $\frac{x_n}{L}$, де $x_n$ - координата максимуму,
    $L$ - відстань до екрана, теж постійне, а отже, при збільшенні довжини
    $L$, координати максимумів будуть збільшуватися. 
    Тобто дифракційна картина буде розширюватись(ширина головних максимумів стане більше). 
    Інтенсивність буде спадати.
    Якщо ж зменшувати $L$, то дифракційна картина звузиться.
    Інтенсивність буде збільшуватися.

    
\end{enumerate}