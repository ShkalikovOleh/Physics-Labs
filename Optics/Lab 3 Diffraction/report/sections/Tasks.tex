\begin{center}
    \Large{\textbf{Практична частина}}
\end{center}

\vspace{1mm}

У результаті виконання роботи, були отримані наступні дослідні дані($L$(см) - відстань до лазера),
$x$ - координата відповідного максимуму

\begin{table}[h!]
    \centering
    \begin{tabular}{|c|c|c|c|}
        \hline
        \textbf{Градка} & \textbf{n} & \textbf{L(см)} & \textbf{x(см)} \\        
        \hline
        \multirow{6}{*}{1} & 1 & \multirow{6}{*}{66.5} & 1.1 \\
        \cline{2-2} \cline{4-4}
         & 2 &  & 2.2 \\
        \cline{2-2} \cline{4-4}
         & 3 &  & 3.3 \\
        \cline{2-2} \cline{4-4}
        & 4 &  & 4.3 \\
        \cline{2-2} \cline{4-4}
        & 5 &  & 5.4 \\
        \cline{2-2} \cline{4-4}
        & 6 &  & 6.5 \\
        \hline

        
        \multirow{2}{*}{2} & 1 & \multirow{2}{*}{21.2} & 3.7 \\        
        \cline{2-2} \cline{4-4}
        & 2 &  & 7.9 \\
        \hline
    \end{tabular}

    \caption{Дослідні дані}
\end{table}

З формули \ref{eq:10} визначимо період градки:

$$ d = \frac{n \lambda}{\sin{\varphi_n}} = \frac{n \lambda}{\sin{\arctan{\frac{x_n}{L}}}} $$

Знаючи значення $\lambda = 632*10^{-9}$(м) отримаємо наступні значення \; періоду 
градки(таблиця 2)

\begin{table}[h]
    \centering
    \begin{tabular}{|c|c|c|c|c|}
        \hline
        \textbf{Градка} & \textbf{n} & $\sin{\varphi}$ & $d$\textbf{(мкм)} & $\langle d \rangle$\textbf{(мкм)} \\        
        \hline
        \multirow{6}{*}{1} & 1 & 0.017  & 38 & \multirow{6}{*}{38.5} \\
        \cline{2-4}
         & 2 & 0.033 & 38  & \\         	                                        
        \cline{2-4}
        & 3 & 0.05 & 38 &\\
        \cline{2-4}
        & 4 & 0.064 & 39 &\\
        \cline{2-4}
        & 5 & 0.081 & 39 &\\
        \cline{2-4}
        & 6 & 0.097 & 39 &\\
        \hline

        \multirow{2}{*}{2} & 1 & 0.172 & 3.7 & \multirow{2}{*}{3.7} \\        
        \cline{2-4}
        & 2 & 0.176 & 3.6  &\\
        \hline
    \end{tabular}

    \caption{Результати обчислень}
\end{table}

Похибки вимірювань обчислимо як стандартне відхилення
середнього арифметичного:

$$ \Delta d = |\langle d \rangle - d_i| $$
$$ \Delta \langle d \rangle = \sqrt{\frac{\sum\limits_{i}^{n} {\Delta d_i}^2}{n(n-1)}} $$

\begin{table}[h]
    \centering
    \begin{tabular}{|c|c|c|c|c|}
        \hline
        \textbf{Градка} & \textbf{n} & $\Delta d$\textbf{(мкм)} & $\Delta \langle d \rangle$\textbf{(мкм)} & $\varepsilon_{\langle D \rangle}\% $ \\        
        \hline
        \multirow{6}{*}{1} & 1 & 0.44  & \multirow{6}{*}{0.17} & \multirow{6}{*}{0.44} \\
        \cline{2-3} 
         & 2 & 0.42 &   & \\         	                                        
         \cline{2-3} 
        & 3 & 0.39 &  &\\
        \cline{2-3} 
        & 4 & 0.52 &  &\\
        \cline{2-3} 
        & 5 & 0.39 &  &\\
        \cline{2-3} 
        & 6 & 0.33 &  &\\
        \hline

        \multirow{2}{*}{2} & 1 & 0.02 & \multirow{2}{*}{0.02} & \multirow{2}{*}{0.54} \\
        \cline{2-3} 
        & 2 & 0.02 &   &\\
        \hline
    \end{tabular}

    \caption{Похибки}
\end{table}

%\begin{figure}[h]    
%    \centering
%    \includegraphics[width=.7\textwidth]{assets/filename }
%    \caption{Підпис}
%\end{figure}